\section{Introduction}

Nous allons dans ce document présenter notre analyse et traitement d'un jeu de données qui nous a été fournis, contenant des informations sur les communes de la région Rhône-Alpes. Nous verrons tout d'abord en détails quelles sont les données que l'on nous a fournis, quelles autres informations nous avons pu leur adjoindre afin de faciliter leur analyse. On se concentrera ensuite sur la technique utilisée et les résultats que nous avons obtenus, étape par étape.

\section{Jeux de données}

\subsection{Jeu original}

Le jeu de données qui nous était fournis nous a servi de base dans notre recherche. Il propose des données démographiques telles que les taux de naissances/décès, des informations sur les revenus, sur le nombre de foyers, et sur la présence ou non de certaines industries. Le tout par commune.

\subsection{Coordonnées GPS IGN}

Afin de mieux visualiser le jeu original dans l'espace, nous lui avons adjoint les coordonnées GPS de chaque commune, nous permettant ainsi de placer sur une carte les communes et de pouvoir comparer leur emplacement avec d'autres de leurs caractéristiques.

\subsection{Recensement INSEE}

Le jeu original manquant de précision quant au nombre de personnes résidant dans chaque commune, nous avons rajouté des informations démographiques provenant d'un recensement de 2008 effectué par l'INSEE et qui nous permettent de connaître le nombre d'habitant total par commune, tout en ajoutant une nouvelle dimension qui est la classification de ces effectifs par tranches d'âge.

\section{Analyse}

\section{Conclusion}